\documentclass{article}

% if you need to pass options to natbib, use, e.g.:
\PassOptionsToPackage{numbers, compress}{natbib}
% before loading neurips_2019

% ready for submission
% \usepackage{neurips_2019}

% to compile a preprint version, e.g., for submission to arXiv, add add the
% [preprint] option:
%     \usepackage[preprint]{neurips_2019}

% to compile a camera-ready version, add the [final] option, e.g.:
\usepackage[schoolpaper]{neurips_2019}

% to avoid loading the natbib package, add option nonatbib:
%     \usepackage[nonatbib]{neurips_2019}

\usepackage[utf8]{inputenc} % allow utf-8 input
\usepackage[T1]{fontenc}    % use 8-bit T1 fonts
\usepackage{hyperref}       % hyperlinks
\usepackage{url}            % simple URL typesetting
\usepackage{booktabs}       % professional-quality tables
\usepackage{amsfonts}       % blackboard math symbols
\usepackage{nicefrac}       % compact symbols for 1/2, etc.
\usepackage{microtype}      % microtypography
\usepackage{graphicx}

\graphicspath{{images/}} 
\bibliographystyle{unsrtnat}

\title{Forecasting daily COVID-19 spread in regions around the world.}

% The \author macro works with any number of authors. There are two commands
% used to separate the names and addresses of multiple authors: \And and \AND.
%
% Using \And between authors leaves it to LaTeX to determine where to break the
% lines. Using \AND forces a line break at that point. So, if LaTeX puts 3 of 4
% authors names on the first line, and the last on the second line, try using
% \AND instead of \And before the third author name.

\author{%
  Frank Lawrence Nii Adoquaye Acquaye\thanks{http://acquayefrank.github.io} \\
  Faculty of Computer Science\\
  National Research University Higher School of Economics\\
  \texttt{fakvey@edu.hse.ru} \\
  % examples of more authors
  \And
  Bense Valerie Caroline \\
  Faculty of Computer Science\\
  National Research University Higher School of Economics\\
   \texttt{vbense@edu.hse.ru} \\
  % \AND
  % Coauthor \\
  % Affiliation \\
  % Address \\
  % \texttt{email} \\
  % \And
  % Coauthor \\
  % Affiliation \\
  % Address \\
  % \texttt{email} \\
  % \And
  % Coauthor \\
  % Affiliation \\
  % Address \\
  % \texttt{email} \\
}

\begin{document}

\maketitle

\begin{abstract}

\end{abstract}

\section{Problem statement:}
The year 2020 will forever be remembered as the year the earth stood still. This is  primarily due to the spread of \href{https://en.wikipedia.org/wiki/Coronavirus_disease_2019}{COVID-19}. As Data Scientists we seek to provide solutions to problems facing humanity and the world at large. In this regard we seek to develop a forecasting model that will predict the daily spread of COVID-19 in regions around the world.  Our model predicts the number of daily new cases in regions around the world in order to help policy makers plan and manage the COVID-19 pandemic.  

\section{Dataset summary and EDA:}

\subsection{Background of dataset:}
The White House Office of Science and Technology Policy (OSTP) pulled together a coalition of research groups and companies (including Kaggle) to prepare the \href{https://www.kaggle.com/allen-institute-for-ai/CORD-19-research-challenge}{COVID-19 Open Research Dataset (CORD-19)} to attempt to address \href{https://www.kaggle.com/allen-institute-for-ai/CORD-19-research-challenge/tasks}{key open scientific questions on COVID-19}. Those questions are drawn from \href{https://www.nationalacademies.org/event/03-11-2020/standing-committee-on-emerging-infectious-diseases-and-21st-century-health-threats-virtual-meeting-1}{National Academies of Sciences, Engineering, and Medicine’s (NASEM)} and the \href{https://www.who.int/blueprint/priority-diseases/key-action/Global_Research_Forum_FINAL_VERSION_for_web_14_feb_2020.pdf?ua=1}{World Health Organization (WHO)}.

\subsection{Data sources:}
The sources of data used in this project can be obtained from \href{https://www.kaggle.com/c/covid19-global-forecasting-week-5/overview}{Kaggle Dataset}

\subsection{Actual data:}
Since the accuracy of such a model is dependent on the freshness of the data, the most up to date data can be found \href{https://www.kaggle.com/c/covid19-global-forecasting-week-5/overview}{here}

\subsection{Actual data used in project:}
In this project we use frozen dataset i.e dataset that has been frozen in time and this dataset can be found \href{https://github.com/acquayefrank/MLDM2020-Project/tree/master/data}{here}

\subsection{Basic exploratory data analysis}

\end{document}
